% Options for packages loaded elsewhere
\PassOptionsToPackage{unicode}{hyperref}
\PassOptionsToPackage{hyphens}{url}
%
\documentclass[
]{article}
\usepackage{amsmath,amssymb}
\usepackage{iftex}
\ifPDFTeX
  \usepackage[T1]{fontenc}
  \usepackage[utf8]{inputenc}
  \usepackage{textcomp} % provide euro and other symbols
\else % if luatex or xetex
  \usepackage{unicode-math} % this also loads fontspec
  \defaultfontfeatures{Scale=MatchLowercase}
  \defaultfontfeatures[\rmfamily]{Ligatures=TeX,Scale=1}
\fi
\usepackage{lmodern}
\ifPDFTeX\else
  % xetex/luatex font selection
\fi
% Use upquote if available, for straight quotes in verbatim environments
\IfFileExists{upquote.sty}{\usepackage{upquote}}{}
\IfFileExists{microtype.sty}{% use microtype if available
  \usepackage[]{microtype}
  \UseMicrotypeSet[protrusion]{basicmath} % disable protrusion for tt fonts
}{}
\makeatletter
\@ifundefined{KOMAClassName}{% if non-KOMA class
  \IfFileExists{parskip.sty}{%
    \usepackage{parskip}
  }{% else
    \setlength{\parindent}{0pt}
    \setlength{\parskip}{6pt plus 2pt minus 1pt}}
}{% if KOMA class
  \KOMAoptions{parskip=half}}
\makeatother
\usepackage{xcolor}
\usepackage[margin=1in]{geometry}
\usepackage{graphicx}
\makeatletter
\def\maxwidth{\ifdim\Gin@nat@width>\linewidth\linewidth\else\Gin@nat@width\fi}
\def\maxheight{\ifdim\Gin@nat@height>\textheight\textheight\else\Gin@nat@height\fi}
\makeatother
% Scale images if necessary, so that they will not overflow the page
% margins by default, and it is still possible to overwrite the defaults
% using explicit options in \includegraphics[width, height, ...]{}
\setkeys{Gin}{width=\maxwidth,height=\maxheight,keepaspectratio}
% Set default figure placement to htbp
\makeatletter
\def\fps@figure{htbp}
\makeatother
\setlength{\emergencystretch}{3em} % prevent overfull lines
\providecommand{\tightlist}{%
  \setlength{\itemsep}{0pt}\setlength{\parskip}{0pt}}
\setcounter{secnumdepth}{-\maxdimen} % remove section numbering
\ifLuaTeX
  \usepackage{selnolig}  % disable illegal ligatures
\fi
\IfFileExists{bookmark.sty}{\usepackage{bookmark}}{\usepackage{hyperref}}
\IfFileExists{xurl.sty}{\usepackage{xurl}}{} % add URL line breaks if available
\urlstyle{same}
\hypersetup{
  pdftitle={Modelos Lineares},
  pdfauthor={Jhonattan Lino},
  hidelinks,
  pdfcreator={LaTeX via pandoc}}

\title{Modelos Lineares}
\author{Jhonattan Lino}
\date{2023-10-12}

\begin{document}
\maketitle

\hypertarget{modelos-lineares}{%
\section{Modelos Lineares}\label{modelos-lineares}}

\hypertarget{definiuxe7uxe3o}{%
\subsubsection{Definição}\label{definiuxe7uxe3o}}

Uma análise de regressão tem como objetivo descrever a relação do valor
médio esperado de uma variável resposta aleatória dado o conjunto de
outras variáveis explicaticas. A regressão linear tem como tem como
objetivo analisar a relação (função) linear entre essas variáveis.

As duas funções de uma análise de regressão são:

\begin{enumerate}
\def\labelenumi{\roman{enumi}.}
\item
  Exploratória: Análise da relação entre variáveis aleatórias, respostas
  e preditoras.
\item
  Preditiva - Previsãoo dos dados futuros não observados em função das
  variáveis preditoras.
\end{enumerate}

O modelos mais simples de uma relação linear entre variáveis é explicita
da seguinte forma:

\hypertarget{y-beta_0-beta_1x}{%
\paragraph{\texorpdfstring{\(y= \beta_0 + \beta_1x\)}{y= \textbackslash beta\_0 + \textbackslash beta\_1x}}\label{y-beta_0-beta_1x}}

*Assumindo que não haja erro

Na presença de erro o modelo deve ser descrito como:

\hypertarget{ybeta_0-beta_1x-varepsilon}{%
\paragraph{\texorpdfstring{\(y=\beta_0 + \beta_1x + \varepsilon\)}{y=\textbackslash beta\_0 + \textbackslash beta\_1x + \textbackslash varepsilon}}\label{ybeta_0-beta_1x-varepsilon}}

Os termos da equação \(\beta_0\) e \(\beta_1\) são os parâmetros que
definem a relação linear entre as duas variáveis enquanto o
\(\varepsilon\) é o termo aleatório do modelos.

\end{document}
